%%
% The BIThesis Template for Undergraduate Thesis
%
% 北京理工大学毕业设计(论文) —— 使用 XeLaTeX 编译
%
% Copyright 2021-2023 BITNP
%
% This work may be distributed and/or modified under the
% conditions of the LaTeX Project Public License, either version 1.3
% of this license or (at your option) any later version.
% The latest version of this license is in
%   http://www.latex-project.org/lppl.txt
% and version 1.3 or later is part of all distributions of LaTeX
% version 2005/12/01 or later.
%
% This work has the LPPL maintenance status `maintained'.
%
% The Current Maintainer of this work is Feng Kaiyu.
%
% Compile with: xelatex -> biber -> xelatex -> xelatex
%%

% !TeX program = xelatex
% !BIB program = biber

\documentclass[type=bachelor_english]{bithesis}

% 此处仅列出常用的配置。全部配置用法请见「bithesis.pdf」手册。
\BITSetup{
  cover = {
    % 封面需要「北京理工大学」字样图片,如无必要请勿修改该项。
    headerImage = images/header.png,
    % 封面标题需要“华文细黑”,如无必要请勿修改该项。
    xiheiFont = STXIHEI.TTF,
    % 本科生盲审要求删去封面,而不是隐藏封面信息。
    hideCoverInPeerReview = true,
    % 修改封面日期
    % date = May 31 2023,
  },
  info = {
    % 想要删除某项封面信息,直接删除该项即可。
    % 想要让某项封面信息留空(但是保留下划线),请传入空白符组成的字符串,如"{~}"。
    % 如需要换行,则用 “\\” 符号分割。
    titleEn = Your Thesis Title,
    school = School of Mechanical Engineering,
    major = Bechelor of Science in Mechanical Engineering,
    author = Feng Kaiyu,
    studentId = 11xxxxxxxx,
    supervisor = Alex Zhang,
    keywords = {北京理工大学;本科生;毕业设计(论文)},
    keywordsEn = {Computer-Aided Design; FEM; CAM},
    % 如果你的毕设为校外毕设,请将下面这一行语句解除注释(删除第一个百分号字符)并填写你的校外毕设导师名字
    % externalSupervisor = 左偏树,
  },
  style = {
    % 保持参考文献的缩进样式与 Word 模板一致。
    % 如果你不需要此样式,请将此行注释掉。
    bibliographyIndent = false,
    % 如无必要请勿修改该项。
    % head = {自定义页眉文字}
  }
}

% 使用 listings 宏包进行代码块使用,并使用了预定义的样式,
% 你也可以选用自己的喜欢的其他宏包,如 minted;
% 然而由于 minted 依赖 Python 的 Pygments 库作为外部依赖,因此出于模板的建议性考虑,我们没有提供 minted 进行代码块书写的示例。
% 但是,我们仍旧非常建议你使用 minted。
\usepackage{listings}

\usepackage[
  backend=biber,
  style=gb7714-2015,
  gbalign=gb7714-2015,
  gbnamefmt=lowercase,
  gbpub=false,
  doi=false,
  url=false,
  eprint=false,
  isbn=false,
]{biblatex}

% 参考文献引用文件位于 misc/ref.bib
\addbibresource{misc/ref.bib}


% 文档开始
\begin{document}

% 标题页面:如无特殊需要,本部分无需改动
% \input{misc/0_cover.tex}
\MakeCover

% 前置页面定义
\frontmatter

\MakeOriginality

% 摘要:在摘要相应的 TeX 文件处进行摘要部分的撰写
%%
% The BIThesis Template for Bachelor Graduation Thesis
%
% 北京理工大学毕业设计(论文) —— 使用 XeLaTeX 编译
%
% Copyright 2021-2023 BITNP
%
% This work may be distributed and/or modified under the
% conditions of the LaTeX Project Public License, either version 1.3
% of this license or (at your option) any later version.
% The latest version of this license is in
%   http://www.latex-project.org/lppl.txt
% and version 1.3 or later is part of all distributions of LaTeX
% version 2005/12/01 or later.
%
% This work has the LPPL maintenance status `maintained'.
%
% The Current Maintainer of this work is Feng Kaiyu.
%
% Compile with: xelatex -> biber -> xelatex -> xelatex
%
% https://github.com/BITNP/BIThesis

\begin{abstract}
  Conventional  product  development  employs  a  design-build-test  philosophy. 
  The sequentially  executed  development  process  often  results  in  prolonged
  lead  times  and elevated product costs. The proposed e-Design paradigm employs 
  IT-enabled technology for product design, including virtual prototyping (VP) to 
  support a cross-functional team in analyzing  product  performance,  reliability,  
  and  manufacturing costs  early  in  product development, and in making quantitative 
  trade-offs for design decision making. Physical prototypes  of  the  product  design  
  are  then  produced  using  the  rapid  prototyping  (RP) technique  and  computer  
  numerical  control  (CNC)  to  support  design  verification  and functional prototyping, respectively.
\end{abstract}

\begin{abstractEn}
  Conventional  product  development  employs  a  design-build-test  philosophy. 
  The sequentially  executed  development  process  often  results  in  prolonged
  lead  times  and elevated product costs. The proposed e-Design paradigm employs 
  IT-enabled technology for product design, including virtual prototyping (VP) to 
  support a cross-functional team in analyzing  product  performance,  reliability,  
  and  manufacturing costs  early  in  product development, and in making quantitative 
  trade-offs for design decision making. Physical prototypes  of  the  product  design  
  are  then  produced  using  the  rapid  prototyping  (RP) technique  and  computer  
  numerical  control  (CNC)  to  support  design  verification  and functional prototyping, respectively.
\end{abstractEn}


\MakeTOC

% 正文开始
\mainmatter

% 第一章
\chapter{Introduction}

Conventional product development is a design-build-test process.
Product performance and reliability assessments depend heavily on physical tests,
which involve fabricating functional prototypes of the product and usually lengthy and expensive physical tests. 
Fabricating prototypes usually involves manufacturing process planning and fixtures and tooling 
for a very small amount of pro-duction. 
The process can be expensive and lengthy, 
especially when a design change is requested to correct problems found in physical tests.

\section{Introduction}

\begin{figure}[htbp]
  \begin{center}
    \includegraphics[width=0.35\textwidth]{example-image}
  \end{center}
  \caption{Comparison of the percentages of customer needs that are revealed for focus groups and interviews as a function of the number of sessions}\label{fig:a}
\end{figure}

In conventional product development, design and manufacturing tend to be disjointed. 
Often, manufacturability of a product is not considered in design. 
Manufacturing issues usually appear when the design is finalized and tests are completed. 
Design defects related to manufacturing in process planning or production 
are usually found too late to be corrected. Consequently, more manufacturing 
procedures are necessary for production, resulting in elevated product cost\cite{cite1}.
With this highly structured and sequential process, the product development cycle tends
to be extended, cost is elevated, and product quality is often compromised to 
avoid further delay. Costs and the number of engineering change requests (ECRs) 
throughout the product development cycle are often proportional according to the pattern 
shown in Figure \ref{fig:a}. 

It is reported that only 8\% of the total product budget is spent for design; however, in 
the early stage, designdetermines 80\% of the lifetime cost of the product (Anderson 1990). 
Realistically, today’s industries will not survive worldwide competition unless they introduce 
new products of better quality, at lower cost, and with shorter lead times. Many approaches 
and concepts have been proposed over the years, all with a common goalto shorten 
the product development cycle, improve product quality, and reduce product cost\parencite{cite2}.


\section{Background}

A number of proposed approaches are along the lines of virtual prototyping,  which 
is  a  simulation-based  method  that  helps  engineers 
understand product behavior and make design decisions in a virtual environment. 
The virtual environment is a computational framework in which the geometric and 
physical properties of products are accurately simulated and represented. A number
of successful virtual prototypes have  been  reported, such as  Boeing’s 777 jetliner,  
General Motors’ locomotive engine, Chrysler’s automotive interior design, 
and the Stockholm Metro’s Car 2000. In addition to virtual prototyping, 
the concurrent engineering (CE) concept and methodology have been studied 
and developed with emphasis on subjects such as product life cycledesign, 
design for X-abilities (DFX), integrated product and process development (IPPD), and Six Sigma.

% 在这里添加第二章、第三章的 TeX 文件的引用
\input{chapters/2_chapter2.tex}
\input{chapters/3_chapter3.tex}

% 后置内容
\backmatter

% 结论:在结论相应的 TeX 文件处进行结论部分的撰写
\input{misc/1_conclusions.tex}
% 参考文献:如无特殊需要,参考文献相应的 TeX 文件无需改动,添加参考文献请使用 BibTeX 的格式
%   添加至 misc/ref.bib 中,并在正文的相应位置使用 \cite{xxj} 的格式引用参考文献
\input{misc/2_references.tex}
% 附录:在附录相应的 TeX 文件处进行附录部分的撰写
\input{misc/3_appendices.tex}
% 致谢:在致谢相应的 TeX 文件处进行致谢部分的撰写
\input{misc/4_acknowledgements.tex}

\end{document}
