%%
% The BIThesis Template for Graduate Thesis
%
% Copyright 2020-2023 Yang Yating, BITNP
%
% This work may be distributed and/or modified under the
% conditions of the LaTeX Project Public License, either version 1.3
% of this license or (at your option) any later version.
% The latest version of this license is in
%   http://www.latex-project.org/lppl.txt
% and version 1.3 or later is part of all distributions of LaTeX
% version 2005/12/01 or later.
%
% This work has the LPPL maintenance status `maintained'.
%
% The Current Maintainer of this work is Feng Kaiyu.
%
% Compile with: xelatex -> biber -> xelatex -> xelatex

% !TeX program = xelatex
% !BIB program = biber

% 硕士论文模板 type=master
% 博士论文模板 type=doctor
% 开启盲审格式 blindPeerReview=true (如:[type=master,blindPeerReview=true])
% 开启双面打印 twoside=true (如:[type=master,twoside=true])
%
% 在 Linux 和 macOS 系统下,由于版权问题,中文字体和 Windows 系统下的字体不同。
% 如果想要获得与 Word 文档相同的效果,你有两个选择:
% 1. 使用 Windows 系统编译最终的论文。
% 2. 自己手动安装中易字库并添加选项:`\documentclass[...,ctex={fontset=windows}]{bithesis}`。
%
% **更多使用说明请参考 bithesis.pdf **

\documentclass[type=master,twoside]{bithesis}

% 此处仅列出常用的配置。全部配置用法请见「bithesis.pdf」手册。
\BITSetup{
  cover = {
    %% 使用以下参数来自定义封面日期
    date = 2022年6月,
    autoWidthPadding = 0.25em,
  },
  info = {
    % 想要删除某项封面信息,直接删除该项即可。
    % 想要让某项封面信息留空(但是保留下划线),请传入空白符组成的字符串,如"{~}"。
    % 如需要换行,则用 “\\” 符号分割。
    classification = TQ028.1,
    UDC = 540,
    title = 形状记忆聚氨酯的合成及其在织物中的应用,
    % 如需覆盖竖排标题,请配置以下选项。
    % 下面的例子展示了如何在竖排标题中使用垂直或者旋转的英文。
    % verticalTitle = {形状记忆聚氨酯{L } {T } {X }的合成 \rotatebox[origin=c]{-90}{Feng Kaiyu} 及其在织物中的应用},
    titleEn = Synthesis and Application on textile of the Shape Memory Polyurethane,
    author = 张三,
    % 如果想要手动控制盲审模式下的隐藏信息,可以使用宏 \SecretInfo{}。使用方式有两种,如:
    % major = \SecretInfo{材料科学与工程} 可以得到 ******* (用等量的替换符号替代)
    % major = \SecretInfo{材料科学与工程}[ABCDEF] 可以得到 ABCDEF (用你自定义的内容替代)
    major = 材料科学与工程,
    school = 材料学院,
    degree = 工学硕士,
    chairman = 王五教授,
    defenseDate = 2022年6月1日,
    supervisor = 李四教授,
    authorEn = San Zhang,
    schoolEn = Materials Science and Engineering,
    supervisorEn = Prof. Si Li,
    chairmanEn = Prof. Wang Wu,
    degreeEn = Master of Philosophy,
    majorEn = Materials Science and Engineering, 
    defenseDateEn = {June, 12th, 2022},
    keywords = {形状记忆;聚氨酯;织物;合成;应用\textcolor{blue}{(硕士一般选3~6个单词或专业术语,博士一般选3~8个单词或专业术语,且中英文关键词必须对应。)}},
    keywordsEn = shape memory properties; polyurethane; textile; synthesis; application,
    % 必要时置于封面右上角,并按照国家规定进行标记。
    % classifiedLevel = 密级\BigStar 保密期限,
  },
  % 在目录页中不显示摘要和主要符号对照表的标题。
  TOC = {
    abstract = false,
    abstractEn = false,
    symbols = false,
  },
  style = {
    pageVerticalAlign = scattered,
    % 开启 Windows 平台下的中易宋体伪粗体。
    % windowsSimSunFakeBold = true,
  },
  publications = {
    % 以下两个选项将影响「攻读学位期间发表论文与研究成果清单」中名称列表的省略阈值。
    % 一般来说,如果你在全部文献中最低排在第四位,建议你将两个值都设置为 4。
    % 更详细的说明请见手册。
    maxbibnames = 3,
    minbibnames = 1,
  },
  % 采用章节标题级别的附录格式
  appendices / chapterLevel = true,
  const = {
    % 由于现存的 Word 模板、旧有 LaTeX 模板与《北京理工大学研究生学位论文撰写规范》的规定不一致,
    % 论文封面的某些字段内容需要用户根据自己的情况进行手动调整。
    % 目前给出的默认值是按照 2018 年发布的《北京理工大学研究生学位论文撰写规范》中的要求进行设置的。
    % 比如注释掉的这一行:将会修改封面中的「申请学位级别」为「申请学位」。
    % info / degree = {申\quad{}请\quad{}学\quad{}位},
    % info / major = {学\quad{}科\hspace{5pt}/\hspace{5pt}类\quad{}别}
  }
}

% 大部分关于参考文献样式的修改,都可以通过此处的选项进行配置。
% 详情请搜索「biblatex-gb7714-2015 文档」进行阅读。
\usepackage[
  defernumbers=true,
  backend=biber,
  style=gb7714-2015,
  gbalign=gb7714-2015,
  gbnamefmt=lowercase,
  gbpub=false,
  gbannote=true,
  gbpunctin=false,
  doi=false,
  url=false,
  eprint=false,
  isbn=false,
]{biblatex}

% 添加参考文献
\addbibresource{reference/main.bib}
% 攻读学位期间发表论文与研究成果清单,详细使用方法见 `chapters/pub.tex`。
\addbibresource{reference/pub.bib}


\usepackage{graphicx}

\begin{document}

% 封面绘制
\MakeCover

% 打印书脊
\MakePaperBack

% 中文信息与英文信息
\MakeTitle

% 论文原创性声明和使用授权
\MakeOriginality

%%%%%%%%%%%%%%%%%%%%%%%%%%%%%%
%% 前置部分
%%%%%%%%%%%%%%%%%%%%%%%%%%%%%%
\frontmatter

% 摘要
\input{./chapters/abstract.tex}

% 制作目录
\MakeTOC

% 插图目录
\listoffigures
% 表格目录
\listoftables

% 主要符号对照表
\input{./misc/0_symbols.tex}

\mainmatter

% 请根据论文内容,按照顺序添加章节。
%%
% The BIThesis Template for Graduate Thesis
%
% Copyright 2020-2023 Yang Yating, BITNP
%
% This work may be distributed and/or modified under the
% conditions of the LaTeX Project Public License, either version 1.3
% of this license or (at your option) any later version.
% The latest version of this license is in
%   http://www.latex-project.org/lppl.txt
% and version 1.3 or later is part of all distributions of LaTeX
% version 2005/12/01 or later.
%
% This work has the LPPL maintenance status `maintained'.
%
% The Current Maintainer of this work is Feng Kaiyu.

\chapter{绪论}

\textcolor{blue}{
  正文包括绪论、论文具体研究内容及结论部分。博士学位论文:一般为6~10万字,其中绪论要求为1万字左右。硕士学位论文:一般为3~5万字,其中绪论要求为0.5万字左右。(外语学科:中文、日文不少于3万字,西文2万字左右。)
}

\textcolor{blue}{
  绪论一般作为第1章。绪论应包括本研究课题的学术背景及其理论与实际意义;本领域的国内外研究进展及成果、存在的不足或有待深入研究的问题;本研究课题的来源及主要研究内容等。
}


\label{chap:intro}
\section{本论文研究的目的和意义}

近年来,随着人们生活水平的不断提高,人们越来越注重周围环境对身体健康的影响。作为服装是人们时时刻刻最贴近的环境,尤其是内衣,对人体健康有很大的影响。由于合时刻刻最贴近的环境,尤其是内衣,对人体健康有很大的影响。由于合成纤维的衣着舒适性、手感性,天然纤维的发展又成为人们关注的一大热点。

……\cite{Takahashi1996Structure,Xia2002Analysis,Jiang1989,Mao2000Motion,Feng1998}

\section{国内外研究现状及发展趋势}
%\label{sec:***} 可标注label

\subsection{形状记忆聚氨酯的形状记忆机理}
%\label{sec:features}

根据文献\parencite{Jiang2005Size},形状记忆聚合物(SMP)是继形状记忆合金后在80年代发展起来的一种新型形状记忆材料。形状记忆高分子材料在常温范围内具有塑料的性质,即刚性、形状稳定恢复性;同时在一定温度下(所谓记忆温度下)具有橡胶的特性,主要表现为材料的可变形性和形变恢复性。即“记忆初始态-固定变形-恢复起始态”的循环。

固定相只有物理交联结构的聚氨酯称为热塑性SMPU,而有化学交联结构称为热固性SMPU。热塑性和热固性形状记忆聚氨酯的形状记忆原理示意图如图\ref{fig:diagram}所示

\begin{figure}[hbt]
 \centering
 \includegraphics[width=0.75\textwidth]{figures/figure1}
 % \caption[这里的文字将会显示在 listoffigure 中]{这里的文字将会显示在正文中}
 \caption{热塑性形状记忆聚氨酯的形状记忆机理示意图}\label{fig:diagram}
\end{figure}


\subsection{形状记忆聚氨酯的研究进展}
%\label{sec:requirements}
首例SMPU是日本Mitsubishi公司开发成功的……。

\subsection{水系聚氨酯及聚氨酯整理剂}

水系聚氨酯的形态对其流动性,成膜性及加工织物的性能有重要影响,一般分为三种类型\cite{Jiang2005Size} ,如表 \ref{tab:category}所示。

\begin{table}[hbt]
  \centering
  \caption{水系聚氨酯分类} \label{tab:category}
  \begin{tabular*}{0.9\textwidth}{@{\extracolsep{\fill}}cccc}
  \toprule
    类别			&水溶型		&胶体分散型		&乳液型 \\
  \midrule
    状态			&溶解$\sim$胶束	&分散		&白浊 \\
    外观			&水溶型		&胶体分散型		&乳液型 \\
    粒径$/\mu m$	&$<0.001$		&$0.001-0.1$		&$>0.1$ \\
    重均分子量	&$1000\sim 10000$	&数千$\sim 20$万 &$>5000$ \\
  \bottomrule
  \end{tabular*}
\end{table}

\subsubsection{四级节标题}

根据需要,也可设四级节标题。

由于它们对纤维织物的浸透性和亲和性不同,因此在纺织品染整加工中的用途也有差别,其中以水溶型和乳液型产品较为常用。另外,水系聚氨酯又有反应性和非反应性之分。虽然它们的共同特点是分子结构中不含异氰酸酯基,但前者是用封闭剂将异氰酸酯基暂时封闭,在纺织品整理时复出。相互交联反应形成三维网状结构而固着在织物表面。
……


%%
% The BIThesis Template for Graduate Thesis
%
% Copyright 2020-2023 Yang Yating, BITNP
%
% This work may be distributed and/or modified under the
% conditions of the LaTeX Project Public License, either version 1.3
% of this license or (at your option) any later version.
% The latest version of this license is in
%   http://www.latex-project.org/lppl.txt
% and version 1.3 or later is part of all distributions of LaTeX
% version 2005/12/01 or later.
%
% This work has the LPPL maintenance status `maintained'.
%
% The Current Maintainer of this work is Feng Kaiyu.
%
% Compile with: xelatex -> biber -> xelatex -> xelatex

\chapter{具体研究内容}

具体研究内容是学位论文的主要部分,是研究结果及其依据的具体表述,是研究能力的集中体现,一般应包括第2章、第3章至结论前一章。具体研究内容应该结构合理,层次清楚,重点突出,文字简练、通顺。可包括以下各方面:研究对象、研究方法、仪器设备、材料原料、实验和观测结果、理论推导、计算方法和编程原理、数据资料和经过加工整理的图表、理论分析、形成的论点和导出的结论等。具体研究内容各章后可有一节“本章小结”(必要时)。

\begin{them}[留数定理]
\label{thm:res}
  假设$U$是复平面上的一个单连通开子集,$a_1,\ldots,a_n$是复平面上有限个点,$f$是定义在$U\backslash \{a_1,\ldots,a_n\}$上的全纯函数,
  如果$\gamma$是一条把$a_1,\ldots,a_n$包围起来的可求长曲线,但不经过任何一个$a_k$,并且其起点与终点重合,那么:

  \begin{equation}
    \label{eq:res}
    \ointop_{\gamma}f(z)\,\mathrm{d}z = 2\pi\mathbf{i}\sum^n_{k=1}\mathrm{I}(\gamma,a_k)\mathrm{Res}(f,a_k)
  \end{equation}

  如果$\gamma$是若尔当曲线,那么$\mathrm{I}(\gamma, a_k)=1$,因此:

  \begin{equation}
    \label{eq:resthm}
    \ointop_{\gamma}f(z)\,\mathrm{d}z = 2\pi\mathbf{i}\sum^n_{k=1}\mathrm{Res}(f,a_k)
  \end{equation}

  在这里,$\mathrm{Res}(f, a_k)$表示$f$在点$a_k$的留数,$\mathrm{I}(\gamma,a_k)$表示$\gamma$关于点$a_k$的卷绕数。
  卷绕数是一个整数,它描述了曲线$\gamma$绕过点$a_k$的次数。如果$\gamma$依逆时针方向绕着$a_k$移动,卷绕数就是一个正数,
  如果$\gamma$根本不绕过$a_k$,卷绕数就是零。
\end{them}

\begin{proof}
  首先,由……

  其次,……

  所以,由\autoref{thm:res}可知……
  \qedhere
\end{proof}
  


\backmatter

% 结论
\input{./misc/1_conclusion.tex}
% 参考文献
%%
% The BIThesis Template for Bachelor Paper Translation
%
% 北京理工大学毕业设计(论文) —— 使用 XeLaTeX 编译
%
% Copyright 2020-2023 BITNP
%
% This work may be distributed and/or modified under the
% conditions of the LaTeX Project Public License, either version 1.3
% of this license or (at your option) any later version.
% The latest version of this license is in
%   http://www.latex-project.org/lppl.txt
% and version 1.3 or later is part of all distributions of LaTeX
% version 2005/12/01 or later.
%
% This work has the LPPL maintenance status `maintained'.
%
% The Current Maintainer of this work is Feng Kaiyu.
%
% Compile with: xelatex -> biber -> xelatex -> xelatex
%%

\begin{bibprint}
  \printbibliography[heading=none]
\end{bibprint}


% 附录
\input{./misc/3_appendices.tex}

% 个人成果
\input{./misc/4_pub.tex}

% 致谢
\input{./misc/5_acknowledgements.tex}
% 个人简介(仅博士生需要此项)
\input{./misc/6_resume.tex}

% 加入目录
\end{document}
