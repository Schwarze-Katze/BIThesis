%%
% The BIThesis Template for Bachelor Graduation Thesis
%
% 北京理工大学毕业设计(论文) —— 使用 XeLaTeX 编译
%
% Copyright 2021-2023 BITNP
%
% This work may be distributed and/or modified under the
% conditions of the LaTeX Project Public License, either version 1.3
% of this license or (at your option) any later version.
% The latest version of this license is in
%   http://www.latex-project.org/lppl.txt
% and version 1.3 or later is part of all distributions of LaTeX
% version 2005/12/01 or later.
%
% This work has the LPPL maintenance status `maintained'.
%
% The Current Maintainer of this work is Feng Kaiyu.
%
% Compile with: xelatex -> biber -> xelatex -> xelatex

% !TeX program = xelatex
% !BIB program = biber


% 开启盲审格式 blindPeerReview=true (如:[type=bachelor,blindPeerReview=true])

\documentclass[type=bachelor]{bithesis}

% 此处仅列出常用的配置。全部配置用法请见「bithesis.pdf」手册。
\BITSetup{
  cover = {
    % 在封面中载入有「北京理工大学」字样的图片,如无必要请勿改动。
    headerImage = images/header.png,
    % 在封面标题中使用思源黑体,使用此选项可以保证与 Word 封面标题的字体一致。
    xiheiFont = STXIHEI.TTF,
    %% 使用以下参数来自定义封面日期
    % date = 2022年6月,
    % 本科生盲审要求删去封面,而不是隐藏封面信息。
    hideCoverInPeerReview = true,
  },
  info = {
    % 想要删除某项封面信息,直接删除该项即可。
    % 想要让某项封面信息留空(但是保留下划线),请传入空白符组成的字符串,如"{~}"。
    % 如需要换行,则用 “\\” 符号分割。
    title = 北京理工大学本科生毕业设计(论文)题目,
    titleEn = {The Subject of Undergraduate Graduation Project (Thesis) of Beijing Institute of Technology},
    school = 计算机学院,
    major = 计算机科学与技术,
    class = 0561xxxx,
    author = 惠计算,
    studentId = 11xxxxxxxx,
    supervisor = 张哈希,
    keywords = {北京理工大学;本科生;毕业设计(论文)},
    keywordsEn = {BIT; Undergraduate; Graduation Project (Thesis)},
    % 如果你的毕设为校外毕设,请将下面这一行语句解除注释(删除第一个百分号字符)并填写你的校外毕设导师名字
    % externalSupervisor = 左偏树,
  },
  style = {
    % 开启 Windows 平台下的中易宋体伪粗体。
    % windowsSimSunFakeBold = true,
  }
}

% 使用 listings 宏包进行代码块使用,并使用了预定义的样式,
% 你也可以选用自己的喜欢的其他宏包,如 minted;
% 然而由于 minted 依赖 Python 的 Pygments 库作为外部依赖,因此出于模板的简洁程度考虑,我们没有提供 minted 进行代码块书写的示例。
\usepackage{listings}


% 大部分关于参考文献样式的修改,都可以通过此处的选项进行配置。
% 详情请搜索「biblatex-gb7714-2015 文档」进行阅读。
\usepackage[
  backend=biber,
  style=gb7714-2015,
  gbalign=gb7714-2015,
  gbnamefmt=lowercase,
  gbpub=false,
  doi=false,
  url=false,
  eprint=false,
  isbn=false,
]{biblatex}

% 参考文献引用文件位于 misc/ref.bib
\addbibresource{misc/ref.bib}

% 如果要按照计算机学院的要求,
% 在外文翻译报告中使用带有“北京理工大学”水印
% 请使用此 issue 提供的方法:
% https://github.com/BITNP/BIThesis/issues/350#issuecomment-1565974141

% 文档开始
\begin{document}

% 标题页面:如无特殊需要,本部分无需改动
% \input{misc/0_cover.tex}
\MakeCover

% 原创性声明:如无特殊需要,本部分无需改动
% 更改为 PDF 页面插入,如需要添加内容,可考虑先用 Word 制作再覆盖 misc/1_originality.pdf
% ====== 原创性声明(PDF 格式)======
\begin{blindPeerReview}
  \includepdf{misc/1_originality.pdf}\newpage
\end{blindPeerReview}
% ====== 原创性声明(PDF 格式)======
% ====== 原创性声明(LaTeX 格式)======
% \input{./misc/1_originality.tex}
% ====== 原创性声明(LaTeX 格式)======

% 前置页面定义
\frontmatter
% 摘要:在摘要相应的 TeX 文件处进行摘要部分的撰写
%%
% The BIThesis Template for Bachelor Graduation Thesis
%
% 北京理工大学毕业设计(论文) —— 使用 XeLaTeX 编译
%
% Copyright 2021-2023 BITNP
%
% This work may be distributed and/or modified under the
% conditions of the LaTeX Project Public License, either version 1.3
% of this license or (at your option) any later version.
% The latest version of this license is in
%   http://www.latex-project.org/lppl.txt
% and version 1.3 or later is part of all distributions of LaTeX
% version 2005/12/01 or later.
%
% This work has the LPPL maintenance status `maintained'.
%
% The Current Maintainer of this work is Feng Kaiyu.
%
% Compile with: xelatex -> biber -> xelatex -> xelatex
%
% https://github.com/BITNP/BIThesis

\begin{abstract}
  Conventional  product  development  employs  a  design-build-test  philosophy. 
  The sequentially  executed  development  process  often  results  in  prolonged
  lead  times  and elevated product costs. The proposed e-Design paradigm employs 
  IT-enabled technology for product design, including virtual prototyping (VP) to 
  support a cross-functional team in analyzing  product  performance,  reliability,  
  and  manufacturing costs  early  in  product development, and in making quantitative 
  trade-offs for design decision making. Physical prototypes  of  the  product  design  
  are  then  produced  using  the  rapid  prototyping  (RP) technique  and  computer  
  numerical  control  (CNC)  to  support  design  verification  and functional prototyping, respectively.
\end{abstract}

\begin{abstractEn}
  Conventional  product  development  employs  a  design-build-test  philosophy. 
  The sequentially  executed  development  process  often  results  in  prolonged
  lead  times  and elevated product costs. The proposed e-Design paradigm employs 
  IT-enabled technology for product design, including virtual prototyping (VP) to 
  support a cross-functional team in analyzing  product  performance,  reliability,  
  and  manufacturing costs  early  in  product development, and in making quantitative 
  trade-offs for design decision making. Physical prototypes  of  the  product  design  
  are  then  produced  using  the  rapid  prototyping  (RP) technique  and  computer  
  numerical  control  (CNC)  to  support  design  verification  and functional prototyping, respectively.
\end{abstractEn}


\MakeTOC

% 正文开始
\mainmatter

% 第一章
\chapter{Introduction}

Conventional product development is a design-build-test process.
Product performance and reliability assessments depend heavily on physical tests,
which involve fabricating functional prototypes of the product and usually lengthy and expensive physical tests. 
Fabricating prototypes usually involves manufacturing process planning and fixtures and tooling 
for a very small amount of pro-duction. 
The process can be expensive and lengthy, 
especially when a design change is requested to correct problems found in physical tests.

\section{Introduction}

\begin{figure}[htbp]
  \begin{center}
    \includegraphics[width=0.35\textwidth]{example-image}
  \end{center}
  \caption{Comparison of the percentages of customer needs that are revealed for focus groups and interviews as a function of the number of sessions}\label{fig:a}
\end{figure}

In conventional product development, design and manufacturing tend to be disjointed. 
Often, manufacturability of a product is not considered in design. 
Manufacturing issues usually appear when the design is finalized and tests are completed. 
Design defects related to manufacturing in process planning or production 
are usually found too late to be corrected. Consequently, more manufacturing 
procedures are necessary for production, resulting in elevated product cost\cite{cite1}.
With this highly structured and sequential process, the product development cycle tends
to be extended, cost is elevated, and product quality is often compromised to 
avoid further delay. Costs and the number of engineering change requests (ECRs) 
throughout the product development cycle are often proportional according to the pattern 
shown in Figure \ref{fig:a}. 

It is reported that only 8\% of the total product budget is spent for design; however, in 
the early stage, designdetermines 80\% of the lifetime cost of the product (Anderson 1990). 
Realistically, today’s industries will not survive worldwide competition unless they introduce 
new products of better quality, at lower cost, and with shorter lead times. Many approaches 
and concepts have been proposed over the years, all with a common goalto shorten 
the product development cycle, improve product quality, and reduce product cost\parencite{cite2}.


\section{Background}

A number of proposed approaches are along the lines of virtual prototyping,  which 
is  a  simulation-based  method  that  helps  engineers 
understand product behavior and make design decisions in a virtual environment. 
The virtual environment is a computational framework in which the geometric and 
physical properties of products are accurately simulated and represented. A number
of successful virtual prototypes have  been  reported, such as  Boeing’s 777 jetliner,  
General Motors’ locomotive engine, Chrysler’s automotive interior design, 
and the Stockholm Metro’s Car 2000. In addition to virtual prototyping, 
the concurrent engineering (CE) concept and methodology have been studied 
and developed with emphasis on subjects such as product life cycledesign, 
design for X-abilities (DFX), integrated product and process development (IPPD), and Six Sigma.

% 在这里添加第二章、第三章……TeX 文件的引用
\input{chapters/2_chapter2.tex}
% \input{chapters/3_chapter3.tex}

% 后置部分
\backmatter

% 结论:在结论相应的 TeX 文件处进行结论部分的撰写
%%
% The BIThesis Template for Bachelor Graduation Thesis
%
% 北京理工大学毕业设计(论文)结论 —— 使用 XeLaTeX 编译
%
% Copyright 2020-2023 BITNP
%
% This work may be distributed and/or modified under the
% conditions of the LaTeX Project Public License, either version 1.3
% of this license or (at your option) any later version.
% The latest version of this license is in
%   http://www.latex-project.org/lppl.txt
% and version 1.3 or later is part of all distributions of LaTeX
% version 2005/12/01 or later.
%
% This work has the LPPL maintenance status `maintained'.
%
% The Current Maintainer of this work is Feng Kaiyu.
%
% Compile with: xelatex -> biber -> xelatex -> xelatex

\begin{conclusion}
  % 结论部分尽量不使用 \subsection 二级标题,只使用 \section 一级标题

  % 这里插入一个参考文献,仅作参考
  本文结论……\cite{张伯伟2002全唐五代诗格会考}。

  \textcolor{blue}{结论作为毕业设计(论文)正文的最后部分单独排写,但不加章号。结论是对整个论文主要结果的总结。在结论中应明确指出本研究的创新点,对其应用前景和社会、经济价值等加以预测和评价,并指出今后进一步在本研究方向进行研究工作的展望与设想。结论部分的撰写应简明扼要,突出创新性。阅后删除此段。}

  \textcolor{blue}{结论正文样式与文章正文相同:宋体、小四;行距:22 磅;间距段前段后均为 0 行。阅后删除此段。}
\end{conclusion}


% 参考文献:如无特殊需要,参考文献相应的 TeX 文件无需改动,添加参考文献请使用 BibTeX 的格式
%   添加至 misc/ref.bib 中,并在正文的相应位置使用 \cite{xxx} 的格式引用参考文献
%%
% The BIThesis Template for Bachelor Graduation Thesis
%
% 北京理工大学毕业设计(论文)参考文献 —— 使用 XeLaTeX 编译
%
% Copyright 2020-2023 BITNP
%
% This work may be distributed and/or modified under the
% conditions of the LaTeX Project Public License, either version 1.3
% of this license or (at your option) any later version.
% The latest version of this license is in
%   http://www.latex-project.org/lppl.txt
% and version 1.3 or later is part of all distributions of LaTeX
% version 2005/12/01 or later.
%
% This work has the LPPL maintenance status `maintained'.
%
% The Current Maintainer of this work is Feng Kaiyu.
%
% Compile with: xelatex -> biber -> xelatex -> xelatex
%
% 如无特殊需要,本页面无需更改

\begin{bibprint}

% -------------------------------- 示例内容(正式使用时请删除) ------------------------------------- %

% 抑制多次调用 \printbibliography 的 warning,只有示例代码会需要此语句。
\BiblatexSplitbibDefernumbersWarningOff

\textcolor{blue}{参考文献书写规范}

\textcolor{blue}{参考国家标准《信息与文献参考文献著录规则》【GB/T 7714—2015】,参考文献书写规范如下:}

\textcolor{blue}{\textbf{1. 文献类型和标识代码}}

\textcolor{blue}{普通图书:M}\qquad\textcolor{blue}{会议录:C}\qquad\textcolor{blue}{汇编:G}\qquad\textcolor{blue}{报纸:N}

\textcolor{blue}{期刊:J}\qquad\textcolor{blue}{学位论文:D}\qquad\textcolor{blue}{报告:R}\qquad\textcolor{blue}{标准:S}

\textcolor{blue}{专利:P}\qquad\textcolor{blue}{数据库:DB}\qquad\textcolor{blue}{计算机程序:CP}\qquad\textcolor{blue}{电子公告:EB}

\textcolor{blue}{档案:A}\qquad\textcolor{blue}{舆图:CM}\qquad\textcolor{blue}{数据集:DS}\qquad\textcolor{blue}{其他:Z}

\textcolor{blue}{\textbf{2. 不同类别文献书写规范要求}}

\textcolor{blue}{\textbf{期刊}}

\noindent\textcolor{blue}{[序号] 主要责任者. 文献题名[J]. 刊名, 出版年份, 卷号(期号): 起止页码. }
\cite{yuFeiJiZongTiDuoXueKeSheJiYouHuaDeXianZhuangYuFaZhanFangXiang2008, Hajela2012Application}

\printbibliography [type=article,heading=none] 

\textcolor{blue}{\textbf{普通图书}}

\noindent\textcolor{blue}{[序号] 主要责任者. 文献题名[M]. 出版地: 出版者, 出版年: 起止页码. }
\cite{张伯伟2002全唐五代诗格会考, OBRIEN1994Aircraft}

\printbibliography [keyword={book},heading=none] 

\textcolor{blue}{\textbf{会议论文集}}

\noindent\textcolor{blue}{[序号] 主要责任者.题名:其他题名信息[C]. 出版地: 出版者, 出版年. }
\cite{雷光春2012}

\printbibliography [type=proceedings,heading=none] 

\textcolor{blue}{\textbf{专著中析出的文献}}

\noindent\textcolor{blue}{[序号] 析出文献主要责任者. 析出题名[M]//专著主要责任者. 专著题名. 出版地: 出版者, 出版年: 起止页码. }
\cite{白书农}

\printbibliography [type=inbook,heading=none] 

\textcolor{blue}{\textbf{学位论文}}

\noindent\textcolor{blue}{[序号] 主要责任者. 文献题名[D]. 保存地: 保存单位, 年份. }
\cite{zhanghesheng, Sobieski}

\printbibliography [keyword={thesis},heading=none] 

\textcolor{blue}{\textbf{报告}}

\noindent\textcolor{blue}{[序号] 主要责任者. 文献题名[R]. 报告地: 报告会主办单位, 年份. }
\cite{fengxiqiao, Sobieszczanski}

\printbibliography [keyword={techreport},heading=none] 

\textcolor{blue}{\textbf{专利文献}}

\noindent\textcolor{blue}{[序号] 专利所有者. 专利题名:专利号[P]. 公告日期或公开日期[引用日期]. 获取和访问路径. 数字对象唯一标识符.}
\cite{jiangxizhou}

\printbibliography [type=patent,heading=none] 

\textcolor{blue}{\textbf{国际、国家标准}}

\noindent\textcolor{blue}{[序号] 主要责任人. 题名: 其他题名信息[S]. 出版地: 出版者, 出版年: 引文页码.}
\cite{GB/T3792.4-2009}

\printbibliography [keyword={standard},heading=none] 

\textcolor{blue}{\textbf{报纸文章}}

\noindent\textcolor{blue}{[序号] 主要责任者. 文献题名[N]. 报纸名, 年(期): 页码. }
\cite{xiexide}

\printbibliography [keyword={newspaper},heading=none] 

\textcolor{blue}{\textbf{电子文献}}

\noindent\textcolor{blue}{[序号] 主要责任者. 电子文献题名[文献类型/载体类型]. (发表或更新日期) [引用日期]. 获取和访问路径. 数字对象唯一标识符. }
\cite{yaoboyuan}

\printbibliography [keyword={online},heading=none] 

\textcolor{blue}{关于参考文献的未尽事项可参考国家标准《信息与文献参考文献著录规则》(GB/T 7714—2015)}

% 在使用时,请删除/注释上方示例内容,并启用下方语句以输出所有的参考文献
% \printbibliography[heading=none]
\end{bibprint}

% 附录:在附录相应的 TeX 文件处进行附录部分的撰写
\input{misc/4_appendix.tex}
% 致谢:在致谢相应的 TeX 文件处进行致谢部分的撰写
\input{misc/5_acknowledgements.tex}

\end{document}
