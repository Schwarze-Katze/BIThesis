\chapter{快速使用指南}
\label{chap:what}

本手册是针对北京理工大学硕士(博士)学位论文~\LaTeX~ 模板\BIThesis{}的快速使用指南。
旨在使对 \LaTeX{} 不熟悉的同学能够快速上手 \BIThesis{} 模板,
以便能够快速生成符合学校格式要求的硕士(博士)学位论文。

本手册试图达成以下目的:
\begin{itemize}
  \item 包含 \BIThesis{} 模板的快速使用说明。
  \item 针对部分 \LaTeX{} 语法的简单介绍。
  \item 针对不需要对 \BIThesis{} 模板进行细致修改的同学的使用指南。
  \item 针对第一次接触 \BIThesis{} 模板甚至 \LaTeX{} 的同学的使用指南。
\end{itemize}

但由于能力有限,本手册并不是:
\begin{itemize}
  \item \LaTeX{} 语法的详细介绍。(TODO)
  \item \BIThesis{} 模板的详细使用、配置的说明。(TODO)
\end{itemize}

\textbf{\BIThesis{}的最新版本位于:}
\begin{center}
  \url{https://github.com/BITNP/BIThesis}
\end{center}
\section{为什么要用 \BIThesis{}}

学位论文通常具有比较严格的格式要求,这是为了方便同行学术交流的起码要
求,同时也是科学研究严谨性的体现。
然而,由于市场各种排版软件混杂,使用者
水平不一,学生对格式的重视程度不够,学生编写标准格式的学位论存在很多问题。
\BIThesis{} 项目旨在提供一套开箱即用的、
符合北京理工大学硕士(博士)学位论文的 LaTeX 模板,以助力高质量的学术写作。
通过 \BIThesis{} 模板,学生可以轻松撰写符合学校格式要求的学位论文,可避免繁琐的论文格式调整,
从而将关注点更多地放在高质量的内容本身。

\section{安装配置环境}

想要使用 \BIThesis{} 模板,需要以下条件:
\begin{itemize}
  \item 一个 \LaTeX{} 发行版。发行版提供了 \LaTeX{} 的编译器和 \LaTeX{} 语法的支持,是将 \LaTeX{} 源码编译成 PDF 文档的必要条件。
  \item 一款 \LaTeX{} 编辑器。一个专业的编辑器提供了 \LaTeX{} 源码的编辑和预览功能。虽然不是必要的,但是使用编辑器可以大大提高 \LaTeX{} 的使用效率。
\end{itemize}

\subsection{发行版选择}

\begin{table}[]
  \centering
\begin{tabular}{@{}cccc@{}}
\toprule
\textbf{操作系统} & Windows & macOS  & Linux   \\ \midrule
\textbf{推荐环境} & TeXLive & MacTeX & TeXLive \\ \midrule
\textbf{发行版自带编辑器} & TeXworks & TeXShop & TeXworks \\ \bottomrule
\end{tabular}
\end{table}

此外,MikTeX 也是一个不错的选择。其优点在于安装包较小,缺点在于需要用户有相关的经验。

可在对应官方网站上下载安装,也可以使用
\textbf{北京理工大学开源软件镜像服务}进行
下载安装。

% Please add the following required packages to your document preamble:
% \usepackage{booktabs}
\begin{table}[]
\begin{tabular}{@{}cll@{}}
\toprule
\textbf{发行版} & \multicolumn{1}{c}{\textbf{官网}}                  & \multicolumn{1}{c}{\textbf{北理镜像站}} \\ \midrule
TeXLive      & \small{\url{https://www.tug.org/texlive/}} &     \tiny{\url{http://mirror.bit.edu.cn/CTAN/systems/texlive/
Images/}}         \\ \midrule
    MacTeX       &  \small\url{https://www.tug.org/mactex/}   & \tiny{\url{http://mirror.bit.edu.cn/CTAN/systems/mac/mactex/}}                                 \\ \midrule
    MikTeX       &  \small\url{https://miktex.org/}             &                              
\end{tabular}
\end{table}



\subsection{编辑器选择}

对于 TexLive 或者 MacTeX,发行版已经自带了编辑器,可直接使用。

如果想要使用其他编辑器如 VSCode 等,可以自行搜索并配置。

\subsection{Overleaf 的使用说明}

Overleaf 是一个在线的 \LaTeX{} 编辑器,可以直接在网页上进行 \LaTeX{} 的编辑和预览。

选用 Overleaf 有优点也有缺点:
\begin{itemize}
  \item 优点在于:
    \begin{itemize}
      \item 不需要安装 \LaTeX{} 发行版,不需要配置编辑器,直接在网页上进行 \LaTeX{} 的编辑和预览。
      \item 数据保存在云端,可以在多个设备上进行编辑和预览。
      \item 可以共享项目,方便多人协作。(对于毕业论文来说,这个优点并不是很重要)
    \end{itemize}
  \item 缺点在于:
    \begin{itemize}
      \item 由于 Overleaf 是一个在线的编辑器,所以需要保持网络连接,否则无法进行编辑和预览。
      \item 很多同学使用了第三方的文献管理软件,如 Zotero。Overleaf 无法直接与这些软件进行集成,需要手动导入文献。
      \item 网页版的编辑器功能有限,无法进行复杂的自定义配置。
    \end{itemize}
\end{itemize}

因此,需要使用者根据自己的需求进行选择。

此外,我们的模板上传到 Overleaf 以后,需要修改项目配置才能正常编译:
\begin{enumerate}
  \item 选择左上角 \textbf{Menu} 以打开侧边栏。
  \item 修改 \textbf{Settings} 中 \textbf{Compiler} 一项的值为 \textbf{XeLaTeX}(默认为 \textbf{pdfLaTeX})。
\end{enumerate}

% TODO: 提供了 Overleaf 的模板。
\section{快速使用}

在发行版和编译器安装完成后,就可以使用 \BIThesis{}模板了。

\subsection{配置编辑器(以 TeXstudio)为例}

想要通过编辑器直接进行编译和预览的话,还需要进行一些配置。

简单来说,从 \LaTeX{} 源码到 PDF 文档的编译过程如下:
\begin{enumerate}
  \item 编辑器将 \LaTeX{} 源码编译成 \LaTeX{} 语法的中间文件。
  \item 编译器对中间文件进行进一步处理(如交叉引用、目录、索引等)。
  \item 编译器将中间文件编译成 PDF 文档。
\end{enumerate}

因此完成流程为  \hologo{XeLaTeX} $\rightarrow$ \hologo{biber} $\rightarrow$ \hologo{XeLaTeX} $\rightarrow$ \hologo{XeLaTeX} 。其命令行执行的命令为:

\begin{verbatim}
# 第一步 xelatex
xelatex -no-pdf --interaction=nonstopmode main
# 第二步 biber
biber main
# 第三步 xelatex
xelatex -no-pdf --interaction=nonstopmode main
# 第四步 xelatex
xelatex --interaction=nonstopmode

\end{verbatim}

\subsection{完成第一次编译}

\section{模板说明与 LaTeX 学习资料}

