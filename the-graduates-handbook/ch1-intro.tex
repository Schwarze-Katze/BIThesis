\chapter{快速使用指南}
\label{chap:what}

本手册是针对北京理工大学硕士(博士)学位论文~\LaTeX~ 模板\BIThesis{}的快速使用指南。
旨在使对 \LaTeX{} 不熟悉的同学能够快速上手 \BIThesis{} 模板,
以便能够快速生成符合学校格式要求的硕士(博士)学位论文。

本手册试图达成以下目的:
\begin{itemize}
  \item 包含 \BIThesis{} 模板的快速使用说明。
  \item 针对部分 \LaTeX{} 语法的简单介绍。
  \item 针对不需要对 \BIThesis{} 模板进行细致修改的同学的使用指南。
  \item 针对第一次接触 \BIThesis{} 模板甚至 \LaTeX{} 的同学的使用指南。
\end{itemize}

但由于能力有限,本手册并不是:
\begin{itemize}
  \item \LaTeX{} 语法的详细介绍。(TODO)
  \item \BIThesis{} 模板的详细使用、配置的说明。(TODO)
\end{itemize}

\textbf{\BIThesis{}的最新版本位于:}
\begin{center}
  \url{https://github.com/BITNP/BIThesis}
\end{center}
\section{为什么要用 \BIThesis{}}

学位论文通常具有比较严格的格式要求,这是为了方便同行学术交流的起码要
求,同时也是科学研究严谨性的体现。
然而,由于市场各种排版软件混杂,使用者
水平不一,学生对格式的重视程度不够,学生编写标准格式的学位论存在很多问题。
\BIThesis{} 项目旨在提供一套开箱即用的、
符合北京理工大学硕士(博士)学位论文的 LaTeX 模板,以助力高质量的学术写作。
通过 \BIThesis{} 模板,学生可以轻松撰写符合学校格式要求的学位论文,可避免繁琐的论文格式调整,
从而将关注点更多地放在高质量的内容本身。

\section{安装配置环境}

想要使用 \BIThesis{} 模板,需要以下条件:
\begin{itemize}
  \item 一个 \LaTeX{} 发行版。发行版提供了 \LaTeX{} 的编译器和 \LaTeX{} 语法的支持,是将 \LaTeX{} 源码编译成 PDF 文档的必要条件。
  \item 一款 \LaTeX{} 编辑器。一个专业的编辑器提供了 \LaTeX{} 源码的编辑和预览功能。虽然不是必要的,但是使用编辑器可以大大提高 \LaTeX{} 的使用效率。
\end{itemize}

\subsection{发行版选择}

\begin{table}[]
  \centering
\begin{tabular}{@{}cccc@{}}
\toprule
\textbf{操作系统} & Windows & macOS  & Linux   \\ \midrule
\textbf{推荐环境} & TeXLive & MacTeX & TeXLive \\ \midrule
\textbf{发行版自带编辑器} & TeXworks & TeXShop & TeXworks \\ \bottomrule
\end{tabular}
\end{table}

此外,MikTeX 也是一个不错的选择。其优点在于安装包较小,缺点在于需要用户有相关的经验。

可在对应官方网站上下载安装,也可以使用
\textbf{北京理工大学开源软件镜像服务}进行
下载安装。

% Please add the following required packages to your document preamble:
% \usepackage{booktabs}
\begin{table}[]
\begin{tabular}{@{}cll@{}}
\toprule
\textbf{发行版} & \multicolumn{1}{c}{\textbf{官网}}                  & \multicolumn{1}{c}{\textbf{北理镜像站}} \\ \midrule
TeXLive      & \small{\url{https://www.tug.org/texlive/}} &     \tiny{\url{http://mirror.bit.edu.cn/CTAN/systems/texlive/
Images/}}         \\ \midrule
    MacTeX       &  \small\url{https://www.tug.org/mactex/}   & \tiny{\url{http://mirror.bit.edu.cn/CTAN/systems/mac/mactex/}}                                 \\ \midrule
    MikTeX       &  \small\url{https://miktex.org/}             &                              
\end{tabular}
\end{table}



\subsection{编辑器选择}

对于 TexLive 或者 MacTeX,发行版已经自带了编辑器,可直接使用。

如果想要使用其他编辑器如 VSCode 等,可以自行搜索并配置。

\subsection{Overleaf 的使用说明}

Overleaf 是一个在线的 \LaTeX{} 编辑器,可以直接在网页上进行 \LaTeX{} 的编辑和预览。

选用 Overleaf 有优点也有缺点:
\begin{itemize}
  \item 优点在于:
    \begin{itemize}
      \item 不需要安装 \LaTeX{} 发行版,不需要配置编辑器,直接在网页上进行 \LaTeX{} 的编辑和预览。
      \item 数据保存在云端,可以在多个设备上进行编辑和预览。
      \item 可以共享项目,方便多人协作。(对于毕业论文来说,这个优点并不是很重要)
    \end{itemize}
  \item 缺点在于:
    \begin{itemize}
      \item 由于 Overleaf 是一个在线的编辑器,所以需要保持网络连接,否则无法进行编辑和预览。
      \item 很多同学使用了第三方的文献管理软件,如 Zotero。Overleaf 无法直接与这些软件进行集成,需要手动导入文献。
      \item 网页版的编辑器功能有限,无法进行复杂的自定义配置。
    \end{itemize}
\end{itemize}

因此,需要使用者根据自己的需求进行选择。

TODO:
此外,我们的模板上传到 Overleaf 以后

\section{快速使用}

\section{模板说明与 LaTeX 学习资料}

test

\chapter{模板组成与使用}

\chapter{公式、图像和表格}

%
% 本手册也使用BIT-thesis生成,可查看相应\href{https://github.com/BIT-thesis/LaTeX-template/tree/master/BIT-thesis-manual}{手册生成源码}作为参考。
%
% \section{为什么要用BIT-Thesis}
% \label{sec:why}
% 学位论文通常具有比较严格的格式要求,这是为了方便同行学术交流的起码要求,同时也是科学研究严谨性的体现。然而,由于市场各种排版软件混杂,使用者水平不一,学生对格式的重视程度不够,学生编写标准格式的学位论存在很多问题。BIT-Thesis 为符合北京理工大学硕士(博士)学位论文的LaTeX模板。通过BIT-Thesis模板可以轻松撰写符合学校格式要求的学位论文,学生可避免繁琐的论文格式调整,从而将关注点更多地放在高质量的内容本身。
%
% \section{安装配置环境}
% \label{sec:requirements}
%
% \TeX 发展至今日,拥有了众多的发行版。发行版软件合集中包括了各种引擎的可执行程序,以及一些文档类、模板、字体文件、辅助程序等等。
%
% \begin{itemize}
% \item windows、linux用户推荐安装TeX Live套装,并更新宏包(linux系统由于版权问题,未能预装宋体等Windows下的字体,需要手动安装)
% \item OSX用户推荐安装Mac TeX
% \item 由于CTeX套装所含宏包比较陈旧,可能会导致编译无法通过,故不推荐安装。如果已安装CTeX,\textbf{建议将其卸除}。
% \end{itemize}
%
% 可在对应官方网站上下载安装,也可以使用\textbf{北京理工大学开源软件镜像服务}进行下载(推荐): \url{http://mirror.bit.edu.cn/CTAN},其中
% \begin{itemize}
% \item TeX Live镜像:\url{http://mirror.bit.edu.cn/CTAN/systems/texlive/Images/}
% \item Mac TeX镜像:\url{http://mirror.bit.edu.cn/CTAN/systems/mac/mactex/}
% \end{itemize}
%
% 如若希望轻量级的编译环境,可安装MiKTeX。具体安装步骤请参考\href{https://github.com/BIT-thesis/LaTeX-materials}{LaTeX学习资料}目录下的《\href{https://github.com/BIT-thesis/LaTeX-materials/raw/master/Miktex%E5%AE%89%E8%A3%85%E4%B8%8E%E9%85%8D%E7%BD%AE.pdf}{Miktex安装与配置}》。 
%
% \section{编辑器选择}
% 安装完成后,便可对\TeX 进行编写以生成相应论文,也可根据使用习惯使用Texmaker、TeXstudio、Winedit等其他的编辑器进行编写。
%
% 其中,Texmaker下载地址:
%
% \url{http://www.xm1math.net/texmaker/}
%
% \section{快速使用}
% \label{sec:process}
%
% 安装完TeX Live套装(或Mac TeX)后,一般而言所需的环境就配置好了。
%
% 进入BIT-thesis-template-grd文件夹。
% windows系统点击运行BIT-thesis-run.bat脚本,linux系统以及mac系统请点击运行BIT-thesis-run.sh脚本。脚本会自动运行如图\ref{fig:run} 所示(第一次运行可能需要较长时间,请耐心等待)。打开生成的pdf文档demo.pdf查看模板生成内容。
%  
% \begin{figure}[!htp]
%   \centering
%   \includegraphics[width=0.85\textwidth]{figures/BIT-thesis-run}
%   \caption{BIT-thesis-run.bat脚本运行}
%   \label{fig:run}
% \end{figure}
%
%
% \begin{figure}[!htp]
%   \centering
%   {\includegraphics[width=0.85\textwidth]{figures/demo_context}}
%   \caption{生成文档demo.pdf的目录}
%   \label{fig:demo_context}
% \end{figure}
%
% 本模板使用~\XeTeX~ 引擎提供的~\XeLaTeX~的命令处理,作用于“主控文档”demo.tex。
% 若使用\textbf{硕士论文模板},请在~demo.tex~中~\verb|\documentclass|~命令采用~master~选项;若使用\textbf{博士论文模板},请使用~doctor~选项。
% 由于该模板使用~{{\sc Bib}\TeX}~处理参考文献,构建流程为 \textbf{XeLaTeX$\rightarrow$ BibTeX $\rightarrow$ XeLaTeX$\rightarrow$ XeLaTeX}。完整的处理流程是:
%
% {\color{blue}
% \begin{enumerate}
% \item[] ~\verb|xelatex -no-pdf --interaction=nonstopmode demo|
% \item[] ~\verb|bibtex demo| 
% \item[] ~\verb|xelatex -no-pdf --interaction=nonstopmode demo|
% \item[] ~\verb|xelatex --interaction=nonstopmode demo|
% \end{enumerate}}
%
% 运行bibtex的时候会提示一些错误,可能是~{{\sc Bib}\TeX}~对UTF-8支持不充
% 分,一般不影响最终结果。加入~\verb|--interaction=nonstopmode|~参数是不让错误打断编译过程。
% \XeTeX~ 仍存在一些宏包兼容性问题,而这些错误通常不会影响最终的编译结果。
%
% 为方便使用,处理过程已经写入BIT-thesis-run.sh(for Linux)和BIT-thesis-run.bat(for Windows)批处理文件中。编写完修改完tex文件后,直接运行对应.sh或者.bat文件即可。
%
% 如使用Texmaker等编辑器,可使用自定义命令对文档快速构建:
%
% {\color{blue}
% \begin{enumerate}
% \item[] ~\verb|xelatex -no-pdf -interaction=nonstopmode %.tex ||
% \item[] ~\verb|bibtex % || 
% \item[] ~\verb|xelatex -no-pdf -interaction=nonstopmode %.tex ||
% \item[] ~\verb|xelatex -interaction=nonstopmode %.tex|
% \end{enumerate}}
%
%
% \section{模板说明与TeX介绍}
% \label{sec:features}
%  
% 目前网上有两个版本的北理工~\LaTeX~ 模板“2012大眼小蚂蚁版”和“2016汪卫版”,均以上海交通大学的模为基础。本模板在此两个模板的基础上依据《北京理工大学博士、硕士学位论文撰写规范》修改,进一步完善成熟,使得模板能够被北理工硕士博士广泛使用。希望使用者通过本模板的介绍,能够对~\LaTeX~ 有一定了解。
%
% 这个模板的中文解决方案是~\XeTeX/\LaTeX~ 。参考
% 文献建议使用~BibTeX~管理,可以生成符合国标~GBT7714~风格的参考文献列表。
% 可以直接插入EPS/PDF/JPG/PNG格式的图像。
% 模板在~Windows~和~Linux~下测试通过,更详细的系统要求请参考
% \ref{sec:requirements}。
%
% 硕士模板的格式受~BIT-thesis-run.cls控制,方便模板更新和模板修改。依据《北京理工大学博士、硕士学位论文撰写规范》修改,按照本文档说明使用,即可撰写符合《北京理工大学博士、硕士学位论文撰写规范》的学位论文;
% 在对外观进行细微调整时,只需要更新这两个文件,不需要对.tex源文件做修改。这也给模板更新带来了
% 极大方便。\textbf{一般使用者不需要修改该文件。}
%  
% ~\XeLaTeX~ 对应的~\XeTeX~ 对中文字体的支持更好,允许用户使用操作系统字体来代替TeX的标准字体。所以本模板使用~\XeLaTeX~ 引擎处理中文。要使用这个模板协助你完成研究生学位论文的创作,下面的条件必须满足:
%
% \begin{itemize}
% \item  操作系统字体目录中有中文字体;
% \item  \TeX~系统有~\XeTeX~引擎;
% \item  你有使用~\LaTeX~ 的经验。可参见LaTeX学习资料\ref{chap:material}。
% \end{itemize}
