\chapter{快速使用指南}
\label{chap:what}

本手册是针对北京理工大学硕士(博士)学位论文 \LaTeX{} 模板 \BIThesis{} 的快速使用指南。
旨在使对 \LaTeX{} 不熟悉的同学能够快速上手 \BIThesis{} 模板,
以便能够快速生成符合学校格式要求的硕士(博士)学位论文。本手册试图达成以下目的:
\begin{itemize}[noitemsep]
  \item 包含 \BIThesis{} 模板的快速使用说明。
  \item 针对部分 \LaTeX{} 语法的简单介绍。
  \item 针对不需要对 \BIThesis{} 模板进行细致修改的同学的使用指南。
  \item 针对第一次接触 \BIThesis{} 模板甚至 \LaTeX{} 的同学的使用指南。
\end{itemize}

由于篇幅有限,也是为了让本手册能专注于让同学们快速上手,本手册将不会介绍
\LaTeX{} 的详细语法和本模板的详细配置方法。\textbf{请同学们在此下载 \BIThesis{}:}
\begin{center}
  \url{https://github.com/BITNP/BIThesis}
\end{center}

\section{为什么要用 \LaTeX{} 和 \BIThesis{}?}

学术、学位论文有严格的格式要求。为了更多同学的方便使用,校方一般提供大家更为熟悉
的 Word 模板。虽然 Word 确实是大家最常用的排版工具,但是:
\begin{center}
  \kaishu
  如果你有足够多使用 Word 的经历,一定会体验过「同一份 Word 文档,在不同地方打开
  就变得不同」这样的魔幻现实主义色彩的经历。
\end{center}

\LaTeX{} 是专用于高质量的学术论文排版的排版工具,能让同学们更专注于内容本身,更
自信的排版符合格式要求的学术、学位论文。\BIThesis{} 项目旨在提供一套开箱即用的、
符合北京理工大学硕士(博士)学位论文的\LaTeX{} 模板,以助力高质量的学术写作。通
过 \BIThesis{} 模板,学生可以轻松撰写符合学校格式要求的学位论文,可避免繁琐的论
文格式调整,从而将关注点更多地放在高质量的内容本身。

\section{我该如何开始?}

首先,\LaTeX{} 并不是像 Word 一样的一个开箱即用的软件。\LaTeX{} 本质上是一门用于
排版的「语言」或「语法规则」。我们实际上,是以 \underline{纯文本文件}(以
\texttt{.tex} 结尾的文件)为基础,用这样的一套 \underline{拟定好的标记语法} 来
设定文字的格式,并 \underline{利用一些工具},将其转化为符合格式要求的 PDF 文档。

我们重新回顾一下这句话:

\begin{itemize}[noitemsep]
  \item \textbf{\underline{纯文本文件}} 意味着我们只需要创建一个以 \texttt{.tex}
  结尾的文件,即可开始论文内容的撰写;
  \item \textbf{\underline{拟定好的语法}} 则需要我们了解一些 \LaTeX{} 中常用的语
  法语言规则,用来以纯文本的形式描述内容的格式,从而让下面提到的工具可以根据格式
  需要,将文档转化为PDF。
  \item \textbf{\underline{利用一些工具}} 也就表示我们需要这些工具(程序),来将
  纯文本内容转化为符合格式的 PDF 文档:我们或是下载安装他们到本地,或是使用在线
  平台开箱即用;
\end{itemize}

因此,本手册也将以这样的逻辑,为大家分别介绍每处需要的知识 --- 我们将首先介绍如
何「安装这些工具」,并如何更舒服的创建、编写此「纯文本文件」(在自己的电脑上和使
用在线的编辑器是不一样的);而后,我们将在后续的章节,简单的讲述常用的「拟定好的
语法」--- 以让大家快速上手,使用 \BIThesis{} 撰写自己的毕业论文。

\section{在自己的电脑上编写论文}

在这里,我们将在自己的电脑上配置安装撰写 \LaTeX{} 的相关工具。首先,我们搞定
\underline{一些工具} 的安装,来更方便的撰写 \underline{纯文本文件} 并将其转化为
符合格式的 PDF 文档。

\paragraph{一些工具的安装} 在 \LaTeX{} 的世界中,我们的「一些工具」包括将
\LaTeX{} 源码按照格式转换为 PDF 文档的「编译器」,和支撑部分 \LaTeX{} 格式语法的
「宏包」。我们将他们统称为一个 \LaTeX{} 发行版 --- 也就是我们需要在自己的电脑上
安装的软件。

根据同学们使用的操作系统,可以安装相应的 \LaTeX{} 发行版:

\begin{itemize}[noitemsep]
  \item \textbf{Windows 或 Linux}:下载安装 \TeX{}Live;
  \item \textbf{macOS}:下载安装 Mac\TeX{};
\end{itemize}

可在对应官方网站上下载安装,也可以使用\textbf{北京理工大学开源软件镜像服务}进行
下载安装,具体下载链接参考表~\ref{tab:download-latex}。

\begin{table}[htbp]
\centering
\resizebox{\textwidth}{!}{
  \begin{tabular}{lll}
  \toprule
  \textbf{发行版} & \textbf{官网}                  & \textbf{北理镜像站} \\ \midrule
  \TeX{}Live      & \url{https://www.tug.org/texlive/} &     \url{http://mirror.bit.edu.cn/CTAN/systems/texlive/Images/}         \\ \midrule
  Mac\TeX       &  \url{https://www.tug.org/mactex/}   & \url{http://mirror.bit.edu.cn/CTAN/systems/mac/mactex/}                                 \\ \midrule
  % MikTeX       &  \url{https://miktex.org/}             & \\ \bottomrule
\end{tabular}}
\caption{各操作系统下载 \LaTeX{} 发行版位置。}
\label{tab:download-latex}
\end{table}

\paragraph{纯文本文件} 纯文本文件的撰写,
% TODO: Continue writing from here. (Spencer)

一款 \LaTeX{} 编辑器。一个专业的编辑器提供了 \LaTeX{} 源码的编辑和预览功能。虽然
不是必要的,但是使用编辑器可以大大提高 \LaTeX{} 的使用效率。

\subsection{编辑器选择}

对于 TeXLive 或者 MacTeX,发行版已经自带了编辑器,可直接使用。

如果想要使用其他编辑器如 VS Code 等,可以自行搜索并配置。

\section{在线使用 Overleaf 开箱即用的编写论文}

Overleaf 是一个在线的 \LaTeX{} 编辑器,可以直接在网页上进行 \LaTeX{} 的编辑和预览。

选用 Overleaf 有优点也有缺点:
\begin{itemize}
  \item 优点在于:
    \begin{itemize}
      \item 不需要安装 \LaTeX{} 发行版,不需要配置编辑器,直接在网页上进行 \LaTeX{} 的编辑和预览。
      \item 数据保存在云端,可以在多个设备上进行编辑和预览。
      \item 可以共享项目,方便多人协作。(对于毕业论文来说,这个优点并不是很重要)
    \end{itemize}
  \item 缺点在于:
    \begin{itemize}
      \item 由于 Overleaf 是一个在线的编辑器,所以需要保持网络连接,否则无法进行编辑和预览。
      \item 很多同学使用了第三方的文献管理软件,如 Zotero。Overleaf 无法直接与这些软件进行集成,需要手动导入文献。
      \item 网页版的编辑器功能有限,无法进行复杂的自定义配置。
    \end{itemize}
\end{itemize}

因此,需要使用者根据自己的需求进行选择。

此外,我们的模板上传到 Overleaf 以后,需要修改项目配置才能正常编译:
\begin{enumerate}
  \item 选择左上角 \textbf{Menu} 以打开侧边栏。
  \item 修改 \textbf{Settings} 中 \textbf{Compiler} 一项的值为 \textbf{XeLaTeX}(默认为 \textbf{pdfLaTeX})。
\end{enumerate}

% TODO: 提供了 Overleaf 的模板。
\section{快速使用}

在发行版和编译器安装完成后,就可以使用 \BIThesis{}模板了。

\subsection{配置编辑器(以 TeXstudio)为例}

想要通过编辑器直接进行编译和预览的话,还需要进行一些配置。

简单来说,从 \LaTeX{} 源码到 PDF 文档的编译过程如下:
\begin{enumerate}
  \item 编辑器将 \LaTeX{} 源码编译成 \LaTeX{} 语法的中间文件。
  \item 编译器对中间文件进行进一步处理(如交叉引用、目录、索引等)。
  \item 编译器将中间文件编译成 PDF 文档。
\end{enumerate}

因此完成流程为  \hologo{XeLaTeX} $\rightarrow$ \hologo{biber} $\rightarrow$ \hologo{XeLaTeX} $\rightarrow$ \hologo{XeLaTeX} 。其命令行执行的命令为:

\begin{verbatim}
# 第一步 xelatex
xelatex -no-pdf --interaction=nonstopmode main
# 第二步 biber
biber main
# 第三步 xelatex
xelatex -no-pdf --interaction=nonstopmode main
# 第四步 xelatex
xelatex --interaction=nonstopmode

\end{verbatim}

\subsection{完成第一次编译}

\section{模板说明与 LaTeX 学习资料}
